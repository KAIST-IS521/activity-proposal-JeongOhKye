\documentclass[a4paper, 11pt]{article}

\usepackage{kotex} % Comment this out if you are not using Hangul
\usepackage{fullpage}
\usepackage{hyperref}
\usepackage{amsthm}
\usepackage[numbers,sort&compress]{natbib}

\theoremstyle{definition}
\newtheorem{exercise}{Exercise}

\begin{document}
%%% Header starts
\noindent{\large\textbf{IS-521 Activity Proposal}\hfill
                \textbf{JeongOh Kye}} \\
         {\phantom{} \hfill \textbf{JeongOhKye}} \\
         {\phantom{} \hfill Due Date: April 15, 2017} \\
%%% Header ends

\section{Activity Overview}

Binary 코드 사이에 다른 코드를 삽입 할 경우 빈공간에 코드를 삽입하고 JUMP하는 방식으로 삽입이 가능하다. 하지만 코드를 넣을 공간이 부족하면 다른 섹션을 만들어야 되고, offset등 여러가지 생각할 것이 생긴다. 이러한 Binary Rewriting을 도와 주는 툴을 제작해 본다.

\section{Exercises}

\begin{exercise}

  먼저 ELF 파일의 구조를 익히고 ELF Header의 역할을 익힌다.

\end{exercise}

\begin{exercise}

  Dynamic Library나 PIE가 걸린 파일의 경우 relocation table의 구조를 파악하고 어떤식으로 추가해야 할지 알아본다.

\end{exercise}

\begin{exercise}

  offset등을 자동으로 계산하여 사용자가 쉽게 활용할 수 있는 Binary Rewriter를 만들어 본다.

\end{exercise}

\section{Expected Solutions}

ELF (혹은 PE)에 대해서 임의의 위치에 임의의 코드를 삽입하거나 여러가지 변화를 줄수 있도록 도와주는 Binary rewriting 툴입니다. 

\bibliography{references}
\bibliographystyle{plainnat}

\end{document}
